% \iffalse
\let\negmedspace\undefined
\let\negthickspace\undefined
\documentclass[journal,12pt,twocolumn]{IEEEtran}
\usepackage{cite}
\usepackage{amsmath,amssymb,amsfonts,amsthm}
\usepackage{algorithmic}
\usepackage{graphicx}
\usepackage{textcomp}
\usepackage{xcolor}
\usepackage[justification=centering]{caption}
\usepackage{txfonts}
\usepackage{listings}
\usepackage{enumitem}
\usepackage{mathtools}
\usepackage{gensymb}
\usepackage{comment}
\usepackage[breaklinks=true]{hyperref}
\usepackage{tkz-euclide} 
\usepackage{listings}
\usepackage{gvv}                                        
\def\inputGnumericTable{}                                 
\usepackage[latin1]{inputenc}                                
\usepackage{color}                                            
\usepackage{array}                                            
\usepackage{longtable}                                       
\usepackage{calc}                                             
\usepackage{multirow}                                         
\usepackage{hhline}                                           
\usepackage{ifthen}                                           
\usepackage{lscape}
\usepackage{circuitikz}
\newtheorem{theorem}{Theorem}[section]
\newtheorem{problem}{Problem}
\newtheorem{proposition}{Proposition}[section]
\newtheorem{lemma}{Lemma}[section]
\newtheorem{corollary}[theorem]{Corollary}
\newtheorem{example}{Example}[section]
\newtheorem{definition}[problem]{Definition}
\newcommand{\BEQA}{\begin{eqnarray}}
\newcommand{\EEQA}{\end{eqnarray}}
\newcommand{\define}{\stackrel{\triangle}{=}}
\theoremstyle{remark}
\newtheorem{rem}{Remark}
\begin{document}

\bibliographystyle{IEEEtran}
\vspace{3cm}

\title{GATE-2023 (EE) \\Q 29}
\author{MANOJ KUMAR (EE23BTECH11211)}
\maketitle
\newpage

\bigskip

\renewcommand{\thefigure}{\theenumi}
\renewcommand{\thetable}{\theenumi}
\textbf{Q29:}
 The value of parameters of the circuit shown in the figure are
 \begin{center}
 $R_1=2\ohm$,$R_2=2\ohm$,$R_3=3\ohm$,$L=10 mH$,$C=100\micro$F
 \end{center}
 For time \(t<0\), the circuit is at steady state with the switch $ 'K'$ in closed condition. If the switch is opened at $t=0$, the value of the voltage across the inductor \brak{V_L}
 at $t=0^{+}$ in Volts is \rule{2cm}{0.4pt} (Round off to 1 decimal place). 
 
\begin{circuitikz}
    \draw (0,0) to [R, R=$R_3$] (0,2);
    \draw (0,3) to [switch, o-o, name=$K$] (0,2);
   \draw (0,3)-- (0,4);
   \draw (0,4) -- (4,4);
   \draw (3,0) to [american current source, l=$10\,\text{A,}\text{DC}$] (3,4);
    \draw (4,4) to (4,5) to (5,5) to[R, l=$R_1$] (6,5);
    \draw (6,5) to(7,5) to[L, l=$L$] (8,5) to (9,5);
    \draw (4,4)to (4,3)to (5,3) to[R, l=$R_2$] (6,3);
    \draw (6,3)to (7,3) to [C, l=$C$] (8,3) to(9,3);
    \draw (9,5) --(9,3);
    \draw (9,4) -- (10,4);
    \draw (10,4)-- (10,0);
    \draw(10,0)--(0,0);
\end{circuitikz}
 \end{document}
